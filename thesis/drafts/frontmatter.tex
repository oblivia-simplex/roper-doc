\title{Urschleim in Silicon:\\Return Oriented Program\\Evolution with ROPER}
\author{Olivia Lucca Fraser}

\degree{Master of Computer Science}
\degreeinitial{M.C.Sc.}
\mcs
\faculty{Computer Science}
\dept{Faculty of Computer Science}
\supervisor{Nur Zincr-Heywood}
\supervisor{Malcolm Heywood}
\defencemonth{December}
\defenceyear{2018}

\dedicate{This thesis is dedicated to my children, Kai, Nahni, Sophie, Quin, and Faro, and to Andrea Shepard.}
\frontmatter
% \listofalgorithms

\begin{abstract}
Return-orientated programming (ROP) identifies pieces of a process's
executable memory ending in a return instruction (gadgets), and enlists them as
an instruction set in which a new, ``parasitic'' program can be written,
hijacking the process's control flow. Since gadgets are already present in
executable memory, there is no reliance upon memory being mapped as both
writeable and executable, which lets the ROP program (or ``chain'') bypass the
shellcode attack mitigation known as $w\oplus x$. As such ROP represents one of
the most difficult exploit mechanisms to mitigate. This thesis explores
ROP-chain generation as a domain for \emph{evolutionary computation}. It
describes a system called ROPER (Return-Oriented Program Evolution with ROPER),
designed and implemented by the author, which orchestrates the evolution of
ROP-chains towards declaratively specified objectives. The author goes on to
study the behaviour and ecology of the ROP-chain populations generated by ROPER,
and their responses to various environmental pressures.
Issues of importance include: 1) establishing a robust environment for evolution
to discover ROP solutions, 2) the design of variation operators, 3) emergent
strategies for genomic resilience, and 4) the role
of speciation through fitness sharing. Case studies are performed using four very
different tasks representative of: 1) the functional objective of a bare bones
exploit, 2) a supervised learning task, 3) policy discovery for an agent playing
‘Snake’, and 4) an ``unwinnable'' task in which fitness is gauged randomly, so
that the effects of non-selective pressures in the environment can be studied.
Taken together this work represents the first time that ROP evolution has been
explicitly demonstrated (at least in the public domain), and studied across
a range of tasks.
\end{abstract}

\printnoidxglossaries
\clearpage
\addtocontents{gls}{\protect\baselineskip=0.7\protect\baselineskip}

\begin{acknowledgements}
I would like to thank my supervisors, Nur
Zincir-Heywood and Malcolm Heywood, whose support, encouragement, and expertise
have been indispensable to me. My thanks also go to to John Jacobs at Raytheon
SAS, who saw enough interest and potential in this project to grant it his
agency's financial backing, when it was still in its early stages. And thanks to
all the friends and colleagues who helped the ideas presented in these pages
come to life in conversation. Without any hope of being exhaustive, these
include Andrea Shepard, Meredith Patterson, Pete Wolfendale, Giancarlo Sandoval,
Neha Spellfish, Pompolic, Petra Kendall, Peli Grietzer, Alice Maz, 0xdeadbabe,
Reza Negarestani, Dominic Fox and others in the Greytribe and Special
Circumstances chat, Julien Savoie, Sig Cox, and Chris Watts, Corrie Watts,
Raphael Bronfman-Nadas, Deepthi Rajashekar, Aimee Burrows, Rob Curry, Robert
Smith, and Stephen Kelly in the NIMS laboratory, past and present, Ed Prevost,
Gurjeet Clair, Katie Sexton, Sam Kaplan, Evan Grant, and Nick Miles at Tenable
(again, past and present), and Lilly Chalupowski at GoSecure. Thanks to Rob
Pierce at 2Keys for his encouragement and support for this project. Thanks to Amy
Ireland, Diann Bauer, Helen Hester, Katrina Burch, and Patricia Reed, of Laboria
Cuboniks, to Virginia Barrett of VNS Matrix, and to Angus MacGyver, for being
inexhaustible sources of inspiration. A long-overdue thanks goes to Uwe Petersen
and Valerie Kerruish, of the \emph{Altonaer Stiftung für philosophische
Grundlagenforschung}, for their support on an earlier research on logic and
dialectics, which was aborted due to external circumstances, and of which the
current project remains a distant descendant. And thank you to my long-suffering
parents Marion and Zachary Fraser, for the countless hours they gave up, looking
after my brood, so that I could finish this thesis while they're still young.

This research is supported by Raytheon SAS. The research is conducted
as part of the Dalhousie NIMS Lab at: \url{https://projects.cs.dal.ca/projectx/}
\end{acknowledgements}

\clearpage
\mainmatter